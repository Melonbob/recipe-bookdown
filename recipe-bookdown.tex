\documentclass[openany]{book}
\usepackage{lmodern}
\usepackage{amssymb,amsmath}
\usepackage{ifxetex,ifluatex}
\usepackage{fixltx2e} % provides \textsubscript
\ifnum 0\ifxetex 1\fi\ifluatex 1\fi=0 % if pdftex
  \usepackage[T1]{fontenc}
  \usepackage[utf8]{inputenc}
\else % if luatex or xelatex
  \ifxetex
    \usepackage{mathspec}
  \else
    \usepackage{fontspec}
  \fi
  \defaultfontfeatures{Ligatures=TeX,Scale=MatchLowercase}
\fi
% use upquote if available, for straight quotes in verbatim environments
\IfFileExists{upquote.sty}{\usepackage{upquote}}{}
% use microtype if available
\IfFileExists{microtype.sty}{%
\usepackage[]{microtype}
\UseMicrotypeSet[protrusion]{basicmath} % disable protrusion for tt fonts
}{}
\PassOptionsToPackage{hyphens}{url} % url is loaded by hyperref
\usepackage[unicode=true]{hyperref}
\hypersetup{
            pdftitle={Food I Can Make in Less Than an Hour},
            pdfauthor={Jonathan Emery, et al.},
            pdfborder={0 0 0},
            breaklinks=true}
\urlstyle{same}  % don't use monospace font for urls
\usepackage{natbib}
\bibliographystyle{plainnat}
\usepackage{longtable,booktabs}
% Fix footnotes in tables (requires footnote package)
\IfFileExists{footnote.sty}{\usepackage{footnote}\makesavenoteenv{long table}}{}
\usepackage{graphicx,grffile}
\makeatletter
\def\maxwidth{\ifdim\Gin@nat@width>\linewidth\linewidth\else\Gin@nat@width\fi}
\def\maxheight{\ifdim\Gin@nat@height>\textheight\textheight\else\Gin@nat@height\fi}
\makeatother
% Scale images if necessary, so that they will not overflow the page
% margins by default, and it is still possible to overwrite the defaults
% using explicit options in \includegraphics[width, height, ...]{}
\setkeys{Gin}{width=\maxwidth,height=\maxheight,keepaspectratio}
\IfFileExists{parskip.sty}{%
\usepackage{parskip}
}{% else
\setlength{\parindent}{0pt}
\setlength{\parskip}{6pt plus 2pt minus 1pt}
}
\setlength{\emergencystretch}{3em}  % prevent overfull lines
\providecommand{\tightlist}{%
  \setlength{\itemsep}{0pt}\setlength{\parskip}{0pt}}
\setcounter{secnumdepth}{5}
% Redefines (sub)paragraphs to behave more like sections
\ifx\paragraph\undefined\else
\let\oldparagraph\paragraph
\renewcommand{\paragraph}[1]{\oldparagraph{#1}\mbox{}}
\fi
\ifx\subparagraph\undefined\else
\let\oldsubparagraph\subparagraph
\renewcommand{\subparagraph}[1]{\oldsubparagraph{#1}\mbox{}}
\fi

% set default figure placement to htbp
\makeatletter
\def\fps@figure{htbp}
\makeatother

\usepackage{booktabs}
\usepackage{amsthm}
\makeatletter
\def\thm@space@setup{%
  \thm@preskip=8pt plus 2pt minus 4pt
  \thm@postskip=\thm@preskip
}
\makeatother

\usepackage[margin=1.25in]{geometry}

\usepackage[contents,nonumber,index]{cuisine}
\usepackage{nicefrac}
\usepackage{xcolor}
\renewcommand{\recipestepnumberfont}{\color[HTML]{C59F61}}% must use uppercase letters
\renewcommand*{\recipenumberfont}{\large\bfseries\sffamily} %set recipe number font
\renewcommand*{\recipequantityfont}{\sffamily\bfseries} %set recipe quantity font
\renewcommand*{\recipeunitfont}{\sffamily} %set recipe unit font
\renewcommand*{\recipeingredientfont}{\sffamily} %set recipe ingredient font
\renewcommand*{\recipefreeformfont}{\itshape} %set recipe free-form font
\RecipeWidths{1.25\columnwidth}
{0.2\columnwidth}
{0.05\columnwidth}
{0.4\columnwidth}
{0.1\columnwidth}
{0.1\columnwidth}
%\RecipeWidths{Total recipe width} {Number of servings width} {Step
%number width} {Ingredient width} {Quantity width} {Units width}

\title{Food I Can Make in Less Than an Hour}
\author{Jonathan Emery, et al.}
\date{2020-01-30}

\begin{document}
\maketitle

{
\setcounter{tocdepth}{1}
\tableofcontents
}
\chapter*{Welcome}\label{welcome}
\addcontentsline{toc}{chapter}{Welcome}

This is a book of recipes using the
\href{https://bookdown.org/}{Bookdown} package for making books and
long-form reports.

\section*{Contribute a Recipe}\label{contribute-a-recipe}
\addcontentsline{toc}{section}{Contribute a Recipe}

You can contribute a recipe by following the instructions on the
\href{https://github.com/chrisdaaz/recipe-bookdown}{GitHub repository}.
\textbf{Enjoy!}

\href{./recipe-bookdown.pdf}{Download the Book}

\section*{Contacts}\label{contacts}
\addcontentsline{toc}{section}{Contacts}

\begin{itemize}
\tightlist
\item
  Jonathan Emery:
  \href{mailto:jonathan.emery@northwestern.edu}{\nolinkurl{jonathan.emery@northwestern.edu}}
\item
  Chris diaz:
  \href{mailto:chris-diaz@northwestern.edu}{\nolinkurl{chris-diaz@northwestern.edu}}
\end{itemize}

\section*{Contributors}\label{contributors}
\addcontentsline{toc}{section}{Contributors}

\begin{itemize}
\tightlist
\item
  Anna Luce
\item
  Jonathan Diel
\item
  Shelby Hatch
\item
  Kate Flom Derrick
\item
  Lauren Woods
\item
  Victoria Getis
\item
  Jill Wilson
\item
  Leslie Fischer
\item
  Borchuluun Yadamsuren
\item
  Cecile Sison
\item
  Matthew Tayler
\item
  Becca Greenstein
\item
  Chris Diaz
\item
  Lauren McKeen
\item
  Patrick Fise
\item
  Rohit Ramanathan
\item
  Shelby Hatch
\item
  Pat Fise
\item
  Aaron Grecius
\end{itemize}

\chapter{5 Ingredient Peanut Sauce}\label{ingredient-peanut-sauce}

\emph{from Becca Greenstein}

\begin{recipe}{5 Ingredient Peanut Sauce}{5 servings}{5 minutes}
\ingredient[\fr12]{cups}{salted peanut butter (or almond butter, or sunflower butter)}
\ingredient[2-3]{Tbsp}{gluten free tamari (or soy sauce, or coconut aminos)}
\ingredient[1-2]{Tbsp}{maple syrup (or another sweetener of choice)}
\ingredient[1]{tsp}{chili garlic sauce( or 1 Thai red chili, minced, or \fr14 tsp red pepper flake}
\ingredient[2-3]{Tbsp}{lime juice}

To a medium mixing bowl add (starting with the amount at the lower end of the measurement range where applicable) peanut butter, tamari, maple syrup, lime juice, chili sauce, and whisk to combine.

\ingredient[\fr14]{cups}{water}

Add water a little at a time until a thick but pourable sauce is achieved.

\newstep Taste and adjust seasonings as needed (i.e. maple syrup for sweetness, chili garlic sauce for heat). If too thin, add more nut butter. If too thick, add more water.
\newstep Serve over pad thai, spring rolls, stir fry, etc.

\end{recipe}

\chapter{5 Layer Pie}\label{layer-pie}

\emph{from Kate Flom Derrick}

\begin{recipe}{5 Layer Pie}{8}{2.5 hours}
\freeform Preheat oven to 325 degrees
\ingredient[1]{cup}{flour}
\ingredient[1]{stick}{butter}
\ingredient[\fr12]{cup}{chopped pecans}
First Layer: Mix butter with flour with fork; press down into baking dish. Make crust, then sprinkle pecans on top of crust. Bake for 20-25 minutes.

\ingredient[1]{cup}{sugar}
\ingredient[8]{ounces}{soft cream cheese}
\ingredient[1]{cup}{Cool Whip}
Second Layer: Mix sugar and cream cheese until smooth, then add Cool Whip. Put mixture on top of crust.

\ingredient[1]{package}{chocolate instant pudding}
Third Layer: Make instant pudding as direct on package. Spread on top of second layer.

\ingredient[1]{cup}{Cool Whip}
Fourth Layer: spread Cool Whip on top of third layer. Chill for 2 hours.

\ingredient[\fr12]{cup}{coconut shavings}
Fifth Layer: sprinkle coconut shavings on top of fourth layer. Serve!

\end{recipe}

\chapter{Anna's Banana Bread (from
ATK)}\label{annas-banana-bread-from-atk}

\emph{from Anna Luce}

\begin{recipe}{Anna’s Banana Bread}{10 Servings}{2 hours}

\freeform Be sure to use very ripe, heavily speckled (or even black) bananas in this recipe. This recipe can be made using 5 thawed frozen bananas; since they release a lot of liquid naturally, they can bypass the microwaving in step 2 and go directly into the fine-mesh strainer. Do not use a thawed frozen banana in step 4; it will be too soft to slice. Instead, simply sprinkle the top of the loaf with sugar. The test kitchen’s preferred loaf pan measures 8\fr12 by 4\fr14-inch inches; if you use a 9 by 5-inch loaf pan, start checking for doneness five minutes earlier than advised in the recipe. The texture is best when the loaf is eaten fresh, but it can be stored (cool completely first), covered tightly with plastic wrap, for up to 3 days.

\newstep Adjust oven rack to middle position and heat oven to 350 \0 F. Spray 8\fr12 by 4\fr14-inch loaf pan with nonstick cooking spray.

\ingredient[1\fr34]{cups}{unbleached AP flour}
\ingredient[1]{tsp}{baking soda}
\ingredient[1]{\fr12}{table salt}

Whisk flour, baking soda, and salt together in large bowl.

\ingredient[5 of 6]{large}{very ripe banana}

Place 5 bananas in microwave-safe bowl; cover with plastic wrap and cut several steam vents in plastic with paring knife. Microwave on high power until bananas are soft and have released liquid, about 5 minutes. Transfer bananas to fine-mesh strainer placed over medium bowl and allow to drain, stirring occasionally, 15 minutes (you should have \fr12 to \fr34 cup liquid).

\newstep Transfer liquid to medium saucepan and cook over medium-high heat until reduced to \fr14 cup, about 5 minutes. Remove pan from heat, stir reduced liquid into bananas, and mash with potato masher until fairly smooth. 

\ingredient[8]{T}{unsalted butter, melted and cooled slightly}
\ingredient[2]{large}{eggs}
\ingredient[\fr34]{cup}{packed light brown sugar}
\ingredient[1]{tsp}{vanilla extract}

Whisk in butter, eggs, brown sugar, and vanilla.

\ingredient[\fr12]{cup}{toasted and coarsely chopped walnuts}

Pour banana mixture into flour mixture and stir until just combined with some streaks of flour remaining. Gently fold in walnuts, if using. Scrape batter into prepared pan. 

\ingredient[1 of 6]{large}{very ripe banana}
\ingredient[2]{tsp}{granulated sugar}

Slice remaining banana diagonally into \fr14-inch-thick slices. Shingle banana slices on top of either side of loaf, leaving 1\fr12-inch-wide space down center to ensure even rise. Sprinkle granulated sugar evenly over loaf.

\newstep Bake until toothpick inserted in center of loaf comes out clean, 55 to 75 minutes. Cool bread in pan on wire rack 15 minutes, then remove loaf from pan and continue to cool on wire rack. Serve warm or at room temperature.

\end{recipe}

\chapter{Banana Bread}\label{banana-bread}

\emph{from Jonathan Emery}

\begin{recipe}[BananaBread]{J. Wolff Banana Bread}{1 loaf}{90 min (30 min prep, 60 min bake)}
\freeform This recipe makes one loaf of banana bread. Bake it in a J. Wolff bread baker --- although other stone breadpans should work, presumably.
\freeform\rule{\textwidth}{0.05pt}

\newstep Preheat oven to 350 \degrees F.
\ingredient[2-3]{}{overripe bananas}

Mash bananas in a mixing bowl until relatively smooth.

\ingredient[\fr34]{cup}{sugar}
\ingredient[1]{tsp}{vanilla}
\ingredient[2]{tbs}{oil (canola or olive)}
\ingredient[\fr13]{cup}{milk)}

Beat sugar, vanilla, olive oil, and milk into the mashed bananas.

\ingredient[1\fr12]{cup}{flour}
\ingredient[\fr13]{tsp}{baking soda}
\ingredient[\fr3/4]{tsp}{salt}
\ingredient[\fr3/4]{tbs}{ground cinnamon}
\ingredient[\fr1/4]{tsp}{nutmeg}

Add flour, baking soda, salt, cinnamon, and nutmeg. Gently mix just to incorporate --- do not overmix.

\ingredient[\fr12]{cup}{chocolate chips or walnuts}

Fold in walnuts or chocolate chips (or both).

\ingredient[\fr13]{cup}{boiling water}

Pour in boiling water and mix until batter is smooth. It may be a bit watery.

\newstep

Grease stoneware baker and pour in batter. Cook for 60-70 min (check frequently) or until a knife inserted into the bread comes out clean.
\end{recipe}

\chapter{Brown Butter Rice Crispies
Treats}\label{brown-butter-rice-crispies-treats}

\emph{from Cecile Sison}

\begin{recipe}{Brown Butter Rice Crispies Treats}{12 Portions}{15 minutes}
\freeform Adapted from: https://smittenkitchen.com/2009/11/salted-brown-butter-crispy-treats
\freeform Grease an 8x8 pan, set aside
\freeform\rule{\textwidth}{0.05pt}

\ingredient[6]{cup}{Rice Crispies (or generic puffed rice cereal)}
\ingredient[6]{pinch}{Gourmet Kosher Sea Salt (preferably Redmond Real Salt, pink and coarse)}
Mix together in large bowl with your (cleaned hands) and set aside.

\ingredient[1]{stick}{Unsalted Butter}
On medium low heat, in a 1 4quart or higher pot (I suggest one that has white enamel so you can see when the butter has browned), melt butter while watching carefully. It will melt, then foam, then turn clear golden and finally start to turn brown and smell nutty. Stir frequently with a wooden spoon, scraping the bottom of the pot to make sure nothing sticks (the brown flecks). Don't be impatient and keep watching because there is a smal window from when the butter browns and when it starts to burn. 

\ingredient[10.5]{oz}{Mini Marshmallows}
When there is a good amount of brown flecks, turn the oven to its lowest setting and pour in marshmallows. Immediately start stirring until the marshmallow mixture doesn't look lumpy.
Add crispie/salt mixture into the marshmallow/butter mixture in the pot and begin incorporating (you can turn off the heat at this point if you want but it doesn't usually take more than 30 seconds to blend the crispies with the melted marshmallows).
Once blended, pour the mixture into your greased pan.

\freeform Optional: sprinkle more salt on top
\freeform Let cool/set for at least a half an hour, cut into squares whenever you serve them.

\end{recipe}

\textless{}! -- Type:Dessert --\textgreater{} \textless{}! --
Cook:Cecile --\textgreater{}

\chapter{Buffalo Cauliflower}\label{buffalo-cauliflower}

\emph{from Chris Diaz}

\begin{recipe}{Buffalo Cauliflower}{4 servings}{55 minutes}
\freeform Preheat oven to 450 degrees F.  Line baking sheet with parchment paper or foil.
\ingredient[2]{}{heads of cauliflower}

Remove stem and leaves, cut into bite-sized pieces

\ingredient[\fr34]{cup}{baking flour}
\ingredient[1]{cup}{water}
\ingredient[\fr12]{tsp}{garlic powder}
\ingredient[\fr12]{tsp}{salt}
\ingredient[\fr14]{tsp}{ground black pepper}

Whisk until batter is smooth, toss with cauliflower, bake for 20 to 25 minutes (until lightly browned)

\ingredient[2]{tbsp}{butter}
\ingredient[\fr12]{cup}{Frank's RedHot pepper sauce}
\ingredient[1]{tsp}{honey}

Melt butter in saucepan over medium heat, remove from heat and add pepper sauce and honey. Toss with lightly browned cauliflower. Return cauliflower to oven and bake an additional 10 minutes.

\freeform Allow cauliflower to cool before serving, about 10 minutes.

\end{recipe}

\chapter{Chickpea Salad}\label{chickpea-salad}

\emph{from Lauren McKeen}

\begin{recipe}{Chickpea Salad}{6}{30 hr}
\freeform This recipe is easy and delicious!
\ingredient[\fr12]{cups}{uncooked quinoa}
Boil the quinoa according to package directions.

\ingredient[2]{TBSP}{lemon juice}
\ingredient[1]{}{avocado}
Dice the avocado and put it in a large bowl with the lemon juice.

\ingredient[1]{}{cucumber}
\ingredient[2]{}{green onions}
\ingredient[1]{}{roma tomato}
\ingredient[1]{}{carrot}
\ingredient[1]{}{red bell pepper}
Chop the vegetables and add them into the bowl.

\ingredient[1]{can}{chickpeas}
Drain and rinse the chickpeas and put them in the bowl.

\ingredient[1]{TBSP}{olive oil}
\ingredient[2]{TBSP}{red wine vinegar}
\ingredient[1]{TBSP}{ground cumin}
\ingredient[]{}{salt and pepper to taste}
Mix olive oil, vinegar, and cumin together and put it in the bowl. Mix all the ingredients together thoroughly. Enjoy with friends.
\end{recipe}

\chapter{Cranberry Clafoutis}\label{cranberry-clafoutis}

\emph{from Victoria Getis}

\begin{recipe}{Cranberry Clafoutis}{}{1 hr}
\freeform Heat the oven to 425 degrees. Butter a deep 9 or 10-inch pie plate. Sprinkle it with a tablespoon or so of sugar, then swirl dish to coat evenly. Invert to remove excess.
\ingredient[1]{tablespoon}{butter, for greasing pan}
\ingredient[1]{cup}{sugar}
\ingredient[2]{}{eggs}
\ingredient[1]{cup}{flour}
\ingredient[1]{cup}{half-and-half or whole milk}
\ingredient[Pinch]{}{salt}
\freeform Beat eggs well, then add remaining sugar. Beat until smooth. Add flour, and beat again until smooth. Add the half-and-half or milk and salt, and whisk until smooth.
\ingredient[2]{cups}{cranberries}
\ingredient{Scant cup}{}{walnuts}

\freeform Coarsely chop cranberries and walnuts. If using a food processor, do not overprocess -- just pulse until chopped. (It's very fast.) Put cranberry mixture in pie plate, and pour batter over it.
\ingredient[]{}{Confectioners' sugar}
\freeform Bake for about 30 minutes, or until clafoutis is nicely browned on top and a knife inserted into it comes out clean. Sift some confectioners' sugar over it, and serve warm or at room temperature.


\end{recipe}

\chapter{Farmers Market Pasta}\label{farmers-market-pasta}

\emph{from Jonathan Emery}

\begin{recipe}[FarmersMarketPasta]{Farmers' Market Pasta}{6 Portions}{30 min}
\freeform Grab some stuff from the farmers' market and make this pasta. Eat it hot or cold.
\freeform\rule{\textwidth}{0.05pt}
\ingredient[12]{oz}{cavatappi}

Cook pasta according to package directions for \textit{al dente}. Drain and transfer pasta to large bow.

\ingredient[1 of 4]{tbs}{olive oil}
\ingredient[2]{cup}{multicolored cherry tomatoes}
\ingredient[4]{clove}{garlic}
\ingredient[1]{tsp}{kosher salt}
\ingredient[\fr12]{tsp}{ground black pepper}

Heat 1 tbs oil in a large skillet at medium heat. Add tomatoes, garlic, salt, and pepper. Cook, stirring often, until tomatoes begin to soften (2-3 min).  

\ingredient[1]{}{zucchini}
\ingredient[1]{}{red onion, wedged}

Add zucchini and onion and continue cooking until tomatoes burst and zucchini is almost tender (3-4 min).

\ingredient[1]{cup}{corn}

Finally, add corn and cook, stirring constantly, for 1 additional minute.

\ingredient[2]{cups}{packed arugula}
\ingredient[1]{cup}{fresh basil, torn}
\ingredient[3 of 4]{tbs}{olive oil}

Add tomato mixture to posta. Toss with arugula, basil, and remaining olive oil.

\ingredient[\fr12]{cup}{shaved parmesan}

Top with parmesan cheese and serve warm or cold.


\freeform\rule{\textwidth}{0.05pt}

\freeform\rule{\textwidth}{0.05pt}
\textit{nomnomnomnomnom}!
\end{recipe}

\%From RealSimple

\chapter{Fish Curry Sauce}\label{fish-curry-sauce}

\emph{from Pat Fise}

\begin{recipe}{Fish Curry Sauce}{4}{15 minutes}
\freeform This is a good recipe for adding some flavor to your favorite fish
\ingredient[1]{}{Shallot}
\ingredient[1]{cup}{coconut milk}
\ingredient[2]{tablespoons}{oil}
\ingredient[1.5]{cups}{coconut milk}

Finely dice and saute shallots in the oil. Add curry paste and coconut milk. Bring to a simmer. Add in salt and any desired seasonings to taste.

Add in the fish and cook for 5-7 minutes until the fish is flakey.

\freeform Garnish with cilantro and serve with white wine.
\end{recipe}

\chapter{Gateau Basque}\label{gateau-basque}

\emph{from Matthew Tayler}

\begin{recipe}{Gateau Basque}{8}{2 hours}
\freeform This delicious almond flour, pastry cream cake is adapted from the recipe of Basque-native Gerald Hirigoyenis, owner of the restaurants Fringale and Pastis in San Francisco.

\ingredient[2]{qty.}{egg yolks}
\ingredient[\fr{1}{4}]{cup}{sugar}
\ingredient[3]{tbsp}{flour}
\ingredient[1 \fr{1}{4}]{cups}{milk}
\ingredient[1]{bean}{vanilla}

For the pastry cream, split vanilla bean length-wise. Begin warming milk and vanilla bean in saucepan over low heat to later bring to a boil.
In the bowl of a stand mixer, beat egg yolks and sugar together until frothing. Continue mixing, slowly add in flower and mix until combined. Remove bowl from mixer and set convenient to warming milk. Bring milk to boil and remove vanilla bean. Immediately upon boiling, remove from heat and begin pouring half the milk into the flour, egg, and sugar mixture. Blend with a whisk. Bring the remaining milk to a boil. Immediately upon boiling,  to whisk in the contents of the mixing bowl into the saucepan, and continue stirring over heat for 1 minute. Remove from heat and allow to begin cooling.   

\ingredient[8]{tbsp.}{unsalted butter}
\ingredient[1]{cup}{sugar}
\ingredient[2]{qty.}{egg yolks}
\ingredient[1]{tbsp.}{rum}
\ingredient[2]{tsp.}{almond extract}
\ingredient[2]{tsp.}{vanilla extract}
\ingredient[2]{tsp.}{Pastis liqueur, e.g. Pernod}
\ingredient{pinch}{salt}
\ingredient[1 \fr{1}{2}]{cups}{flour}
\ingredient[\fr{1}{3}]{cup}{ground almonds (in coffee grinder, e.g.)}
\ingredient[1]{tsp.}{baking powder}
\ingredient[1]{qty.}{egg, beaten}

In stand mixer with the paddle attachment, beat the butter and sugar together until well creamed. Add the egg yolks one at a time, and continue beating well after each addition. Add rum, almond extract, pastis, vanilla extract, and salt. Add dry ingredients: flour, almond powder, and baking powder. Ingredients should mix on on low speed to form a dough. Remove bowl from mixer and orm the dough into 2 even balls, cover with plastic wrap, and refrigerate for 1 hour or more.\par
Later, preheat the oven to 350 degrees F. Butter and dust a 9-inch round cake pan with flour. On a flour-dusted work surface, roll balls of dough into 9-inch and 11-inch circles. Take the 11-inch circle, and transfer to the cake pan, and gently press the dough down into the sides of the pan. Take pastry cream prepared in step 1 and spread in an even layer on top of first layer of pastry dough.\par
Drape 9-inch circle over the cake pan, on top of the pastry cream, to form the top layer of the cake. Firmly seal in the filling by pinching the edges together and then trim off any uneven edges. Brush the top of the cake with a beaten egg.\par
Bake until golden brown, 40 to 45 minutes. Set aside to cool for 10 minutes before inverting onto a cooling rack. Turn the cake right side up and let it cool completely.

\freeform Transfer to a serving plate, and serve at room temperature.
\end{recipe}

\chapter{\texorpdfstring{Greatest common divisor of \(a\) and \(b\),
\(a\leq b\)}{Greatest common divisor of a and b, a\textbackslash{}leq b}}\label{greatest-common-divisor-of-a-and-b-aleq-b}

\emph{from Aaron Grecius}

\textbf{Ingredients} Two positive integers \(a\) and \(b\), \(a\leq b\).

\textbf{Cooking Time} Approximately \(\log a\) steps.

\textbf{Attribution} From Grampy Euclid's cookbook.

\textbf{Recipe}

\begin{itemize}
\tightlist
\item
  Write \(b=aq+r\) with \(0\leq r<a\).
\item
  If \(r=0\), \(a\) is \(\gcd(a,b)\).
\item
  Otherwise, discard \(b\), rename \(a\) as \(b\), rename \(r\) as \(a\)
  and return to Step 1.
\end{itemize}

Guten Appetit!

\chapter{Homemade Baked Mac and
Cheese}\label{homemade-baked-mac-and-cheese}

\emph{from Borchuluun Yadamsuren}

\begin{recipe}{Homemade Baked Mac and Cheese}{12 servings}{20 min}
\freeform
Instructions

Preheat oven to 350F. Lightly grease a large 3 qt or 4 qt baking dish and set aside.Combine shredded cheeses in a large bowl and set aside.
Cook the pasta one minute shy of al dente according to the package instructions. Remove from heat, drain, and place in a large bowl.
Drizzle pasta with olive oil and stir to coat pasta. Set aside to cool while preparing cheese sauce.
Melt butter in a deep saucepan, dutch oven, or stock pot.
Whisk in flour over medium heat and continue whisking for about 1 minute until bubbly and golden.
Gradually whisk in the milk and heavy cream until nice and smooth. Continue whisking until you see bubbles on the surface and then continue cooking and whisking for another 2 minutes. Whisk in salt and pepper.
Add two cups of shredded cheese and whisk until smooth. Add another two cups of shredded cheese and continue whisking until creamy and smooth. Sauce should be nice and thick.
Stir in the cooled pasta until combined and pasta is fully coated with the cheese sauce.
Pour half of the mac and cheese into the prepared baking dish. Top with remaining 2 cups of shredded cheese and then the remaining mac and cheese.
In a small bowl, combine panko crumbs, Parmesan cheese, melted butter and paprika. Sprinkle over the top and bake until bubbly and golden brown, about 30 minutes. Serve immediately.

\ingredient[16]{oz}{elbow macaroni, cooked (or other tubular pasta)}
\ingredient[1]{tbsp}{extra virgin olive oil}
\ingredient[6]{tbsp}{ unsalted butter}
\ingredient[1/3]{cup}{all purpose flour}
\ingredient[3]{cups}{whole milk}
\ingredient[1]{cup}{heavy whipping cream}
\ingredient[4]{cups}{sharp cheddar cheese shredded}
\ingredient[2]{cups}{Gruyere cheese shredded}
\ingredient[4]{tbsp}{butter melted}
\ingredient[1/2]{cups}{Parmesan cheese shredded}
\ingredient[1/4]{tbsp}{smoked paprika}
salt and pepper to taste

\freeform Add final notes, such as serving suggestions.
\end{recipe}

\chapter{Moroccan Lamb (or Turkey)
Stew}\label{moroccan-lamb-or-turkey-stew}

\emph{From Jonathan Emery}

\begin{recipe}[MoroccanStew]{Moroccan Lamb (or Turkey) Stew}{4 Portions}{1 hour}
\freeform One of my absolute favorite meals.
\freeform\rule{\textwidth}{0.05pt}
\ingredient[1]{lb}{lamb or turkey}

In a well-oiled pan, saut\'{e} meat until over medium-high heat until cooked through. Set cooked meat aside. Deglaze pan if you like, or if meat was fatty.

\ingredient[2]{cups}{onion}
\ingredient[\fr12]{cup}{carrots}

Add oil to pan. Chop onions vertically and carrots diagonally. Add to pan and saut\'{e} for 4-5 minutes on medium heat, or until soft.

\ingredient[\fr34]{tps}{cumin}
\ingredient[\fr34]{tps}{cinnamon}
\ingredient[\fr12]{tps}{coriander}
\ingredient[\fr14]{tps}{red pepper flakes}

Add coriander, cumin, red pepper flakes, and cinnamon to vegetables, quickly mixing in spices. Saut\'{e} for 30 seconds, stirring constantly.

\ingredient[]{}{Reserved lamb/turkey}
\ingredient[2]{cups}{broth}
\ingredient[\fr12]{cup}{golden raisins}
\ingredient[3]{tbs}{tomato paste}
\ingredient[1\fr12]{tbs}{grated lemon rind}
\ingredient[\fr14]{tps}{salt}
\ingredient[15]{oz}{chick peas (rinsed, drained)}

Add reserved meat, broth, raisins, tomato paste, lemon rind, salt, and chick peas. Bring to a boil, then reduce heat and simmer for 4 minutes (or until mixture thickens). Remove from heat

\ingredient[\fr12]{cup}{chopped, fresh cilantro}
\ingredient[1]{tbs}{fresh lemon juice}

Stir in cilantro and lemon juice.


%\freeform\rule{\textwidth}{0.05pt}\newpage
\freeform\rule{\textwidth}{0.05pt}
%\newstep

%\freeform\rule{\textwidth}{0.05pt}
%Serve with couscous or some grain and fresh bread. Yummy!
\end{recipe}

\chapter{One-Pot Chickpea Tiki
Masala}\label{one-pot-chickpea-tiki-masala}

\emph{From Jonathan Diel}

\begin{recipe}{One-Pot Chickpea Tiki Masala}{4}{40 minute}

\freeform This recipe is one of my go to when I want Idian food pretty quickly. It's very easy to make espicially if using canned beans and tomatoes. Original recipe from https://www.makingthymeforhealth.com/one-pot-chickpea-tiki-masala/

\ingredient[2]{Tbsp}{extra virgin olive oil}
\ingredient[1]{medium}{onion, diced}
\ingredient[1]{tsp}{sea salt}
\ingredient[2]{}{jalepeños, cored and finely chopped}
\ingredient[4]{in}{fresh ginger, minced (approx. 4 Tbsp}
\ingredient[6]{}{garlic cloves, minced, approx 3 Tbsp}
\ingredient[1]{tsp}{garam masala}
\ingredient[1]{tsp}{ground cumin}
\ingredient[1]{tsp}{curry powder}
\ingredient[fr 12]{tsp}{smoked paprika}
\ingredient[3]{Tbps}{tomato paste}
\ingredient[2]{}{(15-oz) cans diced fire roasted tomatoes}
\ingredient[1]{c}{vegetable broth}
\ingredient[1]{}{(15-oz) can chickpeas, drained and rinsed}
\ingredient[1]{c}{unsweetened cocounut milk}
\ingredient[4]{c}{cooked brown rice or naan for serving}

Instructions:
In a large pot, warm the oil over medium heat. Add the onion and the salt, stir and cook for 3 minutes. Add the jalapeño, ginger, and garlic then cook for 2 minutes. Lastly, add the spices (garam masala, cumin, curry, paprika, cayenne) and the tomato paste then stir together and cook for 2 more minutes.

Pour the cans of diced tomatoes with their juices and 1 cup vegetable broth into the pot. Bring to a boil and cook for 10 minutes, stirring intermittently.

Reduce heat to a simmer then stir in the coconut milk and chickpeas. Cook until heated through. Serve warm with brown basmati rice, fresh cilantro, plain yogurt and naan. Leftovers can be stored in an airtight container in the refrigerator for up to 3 days.

\freeform I've tried making this recipe in a hot pot with mixed results. I think it still works better on the stove. Even still, the recipe is very easy to make despite the long list of ingredients.

\end{recipe}

\chapter{Orzo Salad}\label{orzo-salad}

\emph{from Jonathan Emery}

\begin{recipe}{Spinach Orzo Salad}{20 Portions}{30 minutes}
\freeform Good to share as a holiday salad!
\freeform\rule{\textwidth}{0.05pt}
\ingredient[16]{oz}{Orzo pasta}

Cook orzo according to directions on package --- slightly al dente is fine as well. Drain and rise with cold water.

\ingredient[9]{oz}{spinach}
\ingredient[\fr34]{cup}{dried cranberries}

Finely chop dried cranberries (craisins) and spinach. Add to orzo in a large serving bowl.

\ingredient[\fr34]{cup}{feta cheese}
\ingredient[\fr34]{cup}{balsamic vinaigrette}
\ingredient[\fr12]{tsp}{dried basil}
\ingredient[\fr14]{tsp}{white pepper, ground}

Add these ingredients to the mix and toss so that greens are well-coated with the vinaigrette. Flavored vinaigrette is a nice touch (e.g., strawberry). Refrigerate if not serving immediately.

\ingredient[\fr14]{cup}{sunflower seeds}

Toss in sunflower seeds prior to serving.

%\freeform\rule{\textwidth}{0.05pt}\newpage
\freeform\rule{\textwidth}{0.05pt}
%\newstep

\freeform Jon's recipe is the best!!!

%\freeform\rule{\textwidth}{0.05pt}
%Serve with couscous or some grain and fresh bread.
\end{recipe}

\chapter{Peanut Tofu Soup}\label{peanut-tofu-soup}

\emph{from Rohit Ramanathan, Adapted from Rev Soup in Charlottesville,
VA}

\begin{recipe}{Spicy Peanut Tofu Soup}{6-8 Servings}{1.5 Hours}
\freeform This is an adaptation spicy peanut tofu soup served at Revolutionary Soup in Charlottesville, Va.
\ingredient[2]{}{Large Onions, diced}
\ingredient[3]{}{Large Carrots, diced}
\ingredient[2]{}{Medium Jalapenos, seeds removed if desired, and diced}
\ingredient[2]{inches}{ginger, minced}
\ingredient[4]{cloves}{garlic, minced}
\ingredient[2]{bbsp}{Curry Powder, plus more to taste}
\ingredient[14]{oz}{canned diced tomatoes, do not drain}
\ingredient[14]{oz}{canned crushed tomatoes}
\ingredient[1][can]{coconut milk}
\ingredient[1]{cup}{smooth peanut butter, plus more to taste}
\ingredient[1]{package}{firm tofu, drained and pressed to remove moisture, and cut into cubes.}
\ingredient[to taste]{salt}

1. In a large pot or dutch oven, sautee the onions and carrots together in about 1 tbsp oil until they are soft. Season with salt to taste and add curry powder.
2. Using a small blender or mortal and pestle, combine the jalapenos, ginger, and garlic together with a pinch of salt until a coarse paste forms.
3. Add this paste to the pot with the carrots and onions and continue cooking until fragrant.
4. Add diced tomatoes, crushed tomatoes, coconut milk, and peanut butter.  Bring to a boil and reduce to a simmer, stirring to ensure that the peanut butter is completely incorporated.
5. Once the peanut butter is fully incorporated, remove from heat and blend using an immersion blender until smooth.
6. Taste the resulting soup and adjust salt, curry powder, and peanut butter as needed. The soup should be spicy, but not overwhelmingly so, and you should be able to taste the peanut but it should not be sweet.
7. Add tofu to the pot and stir. Bring the soup back to a simmer and cook tofu for 5-10 minutes.
8. Serve immediately with a crusty bread for dipping.

\end{recipe}

\chapter{Spinach-Ricotta Pie}\label{spinach-ricotta-pie}

\emph{from Shelby Hatch}

\begin {recipe} {Spinach-Ricotta Pie} {4-6} {3 hours}
\freeform from The Moosewood Cookbook by Molly Katzen
\freeform Preheat oven to 375
\freeform The Crust
\ingredient [1] {cup} {flour}
\ingredient [\fr 13] {cup} {cold butter}

Cut together 1 cup flour (4/5 white plus 1/5 whole wheat is nice) and 1/3 cup cold butter. Use a pastry cutter or two forks, or a food processor fitted with a steel blade.

\ingredient [3] {Tbsp} {cold buttermilk}

When the mixture is uniformly blended, add about 3 tablespoons cold buttermilk - or enough so that the mixture holds together enough to form a ball.
\newstep

Chill the dough at least one hour.
\freeform The Filling
\ingredient [1] {lb} {ricotta cheese}
\ingredient [3] {} {beaten eggs}
\ingredient [1] {small} {diced onion}
\ingredient [3] {Tbsp} {flour}
\ingredient [\fr 12] {cup} {grated sharp cheese}
\ingredient [1] {dash} {nutmeg}

Mix everything together, blending well. Spread into unbaked pie shell.

\ingredient [1] {cup} {sour cream}
\ingredient [] {} {paprika}

Top with 1 cup sour cream spread to the edges of the crust and a generous application of paprika.
\newstep

Bake in 375 oven for 40-45 minutes. Serve piping hot.

\end {recipe}

\chapter{Creamy Turkey Noodle Soup}\label{creamy-turkey-noodle-soup}

\emph{from Jonathan Emery}

\begin{recipe}{Creamy Turkey Noodle Soup}{6 Servings}{30 minutes} %{Title}{Servings}{Prep time}
\freeform A soup that Mimi makes after Thanksgiving.       %Freeform formats a general comment.
\freeform\rule{\textwidth}{0.05pt}                         %Creates a nice horizontal rule.

%1st step ingredients---------------------------------------
    %\ingredient[quantity]{unit}{ingredient}
    %\Ingredient{ingredient and quantity}
\ingredient[3]{cups}{turkey}

%1st step action
Use leftover Thanksgiving turkey (or bake). Set aside.

%2nd step ingredients--------------------------------------
\ingredient[5]{tbsp}{butter}
\ingredient[2]{}{carrots}
\ingredient[1]{}{shallot}
\ingredient[1]{tbsp}{celery ceeds}

%2nd step action:
Saut\'{e} ingredients in butter for 5 minutes.

%3rd step ingredients--------------------------------------
\ingredient[\fr13]{cup}{flour} %\fr means fraction: so \fr13 = 1/3
\ingredient[32]{oz}{chicken broth}
\ingredient[\fr12]{cup}{half and half}
\ingredient[\fr12]{cup}{milk}

%3rd step action:
First stir in flour to saut\'{e}d vegetebles and then add liquids and bring mixture to a boil.

%4th step ingredients--------------------------------------
\ingredient[6]{oz}{Reames brand noodles}

%4th step action:
Add dry noodles to soup and simmer for about 20 minutes.

Cook orzo according to directions on package --- slightly al dente is fine as well. Drain and rise with cold water.

%5th step ingredients--------------------------------------
\ingredient[\fr14]{tsp}{salt}
\ingredient[\fr14]{tsp}{pepper}

Add salt and pepper to taste.

\freeform\rule{\textwidth}{0.05pt}

\freeform Good to eat for... weeks?

\end{recipe}

\chapter{Yorkshire Pudding}\label{yorkshire-pudding}

\emph{from Jonathan Emery}

\begin{recipe}{Yorshire Pudding}{4 portions}{1\fr12 hours}

\freeform This recipe was taken from the cuisine package examples of Ben Cohen.

\ingredient[\fr12]{pt}{milk}
\ingredient[2]{oz}{butter}
\ingredient[5]{oz}{self-raising flour}

Heat the milk and butter until nearly boiling. Add flour and allow to seeth over.

\ingredient[3]{}{eggs}
\ingredient{to taste}{salt and pepper}

Add the remaining eggs and whisk again. Cook at 200\0C for about 1 hour.

\end{recipe}

\end{document}
