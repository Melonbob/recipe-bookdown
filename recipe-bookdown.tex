\documentclass[openany]{book}
\usepackage{lmodern}
\usepackage{amssymb,amsmath}
\usepackage{ifxetex,ifluatex}
\usepackage{fixltx2e} % provides \textsubscript
\ifnum 0\ifxetex 1\fi\ifluatex 1\fi=0 % if pdftex
  \usepackage[T1]{fontenc}
  \usepackage[utf8]{inputenc}
\else % if luatex or xelatex
  \ifxetex
    \usepackage{mathspec}
  \else
    \usepackage{fontspec}
  \fi
  \defaultfontfeatures{Ligatures=TeX,Scale=MatchLowercase}
\fi
% use upquote if available, for straight quotes in verbatim environments
\IfFileExists{upquote.sty}{\usepackage{upquote}}{}
% use microtype if available
\IfFileExists{microtype.sty}{%
\usepackage[]{microtype}
\UseMicrotypeSet[protrusion]{basicmath} % disable protrusion for tt fonts
}{}
\PassOptionsToPackage{hyphens}{url} % url is loaded by hyperref
\usepackage[unicode=true]{hyperref}
\hypersetup{
            pdftitle={Food I Can Make in Less Than an Hour},
            pdfauthor={Jonathan Emery},
            pdfborder={0 0 0},
            breaklinks=true}
\urlstyle{same}  % don't use monospace font for urls
\usepackage{natbib}
\bibliographystyle{plainnat}
\usepackage{longtable,booktabs}
% Fix footnotes in tables (requires footnote package)
\IfFileExists{footnote.sty}{\usepackage{footnote}\makesavenoteenv{long table}}{}
\usepackage{graphicx,grffile}
\makeatletter
\def\maxwidth{\ifdim\Gin@nat@width>\linewidth\linewidth\else\Gin@nat@width\fi}
\def\maxheight{\ifdim\Gin@nat@height>\textheight\textheight\else\Gin@nat@height\fi}
\makeatother
% Scale images if necessary, so that they will not overflow the page
% margins by default, and it is still possible to overwrite the defaults
% using explicit options in \includegraphics[width, height, ...]{}
\setkeys{Gin}{width=\maxwidth,height=\maxheight,keepaspectratio}
\IfFileExists{parskip.sty}{%
\usepackage{parskip}
}{% else
\setlength{\parindent}{0pt}
\setlength{\parskip}{6pt plus 2pt minus 1pt}
}
\setlength{\emergencystretch}{3em}  % prevent overfull lines
\providecommand{\tightlist}{%
  \setlength{\itemsep}{0pt}\setlength{\parskip}{0pt}}
\setcounter{secnumdepth}{5}
% Redefines (sub)paragraphs to behave more like sections
\ifx\paragraph\undefined\else
\let\oldparagraph\paragraph
\renewcommand{\paragraph}[1]{\oldparagraph{#1}\mbox{}}
\fi
\ifx\subparagraph\undefined\else
\let\oldsubparagraph\subparagraph
\renewcommand{\subparagraph}[1]{\oldsubparagraph{#1}\mbox{}}
\fi

% set default figure placement to htbp
\makeatletter
\def\fps@figure{htbp}
\makeatother

\usepackage{booktabs}
\usepackage{amsthm}
\makeatletter
\def\thm@space@setup{%
  \thm@preskip=8pt plus 2pt minus 4pt
  \thm@postskip=\thm@preskip
}
\makeatother

\usepackage[margin=1.25in]{geometry}

\usepackage[contents,nonumber,index]{cuisine}
\usepackage{nicefrac}
\usepackage{xcolor}
\renewcommand{\recipestepnumberfont}{\color[HTML]{C59F61}}% must use uppercase letters
\renewcommand*{\recipenumberfont}{\large\bfseries\sffamily} %set recipe number font
\renewcommand*{\recipequantityfont}{\sffamily\bfseries} %set recipe quantity font
\renewcommand*{\recipeunitfont}{\sffamily} %set recipe unit font
\renewcommand*{\recipeingredientfont}{\sffamily} %set recipe ingredient font
\renewcommand*{\recipefreeformfont}{\itshape} %set recipe free-form font
\RecipeWidths{1.25\columnwidth}
{0.2\columnwidth}
{0.05\columnwidth}
{0.4\columnwidth}
{0.1\columnwidth}
{0.1\columnwidth}
%\RecipeWidths{Total recipe width} {Number of servings width} {Step
%number width} {Ingredient width} {Quantity width} {Units width}

\title{Food I Can Make in Less Than an Hour}
\author{Jonathan Emery}
\date{2020-01-29}

\begin{document}
\maketitle

{
\setcounter{tocdepth}{1}
\tableofcontents
}
\chapter*{Welcome}\label{welcome}
\addcontentsline{toc}{chapter}{Welcome}

This is a book of recipes using the
\href{https://bookdown.org/}{Bookdown} package for making books and
long-form reports.

\section*{Contribute a Recipe}\label{contribute-a-recipe}
\addcontentsline{toc}{section}{Contribute a Recipe}

You can contribute a recipe by following the instructions on the
\href{https://github.com/chrisdaaz/recipe-bookdown}{GitHub repository}.
\textbf{Enjoy!}

\href{./recipe-bookdown.pdf}{Download the Book}

\section*{Contacts}\label{contacts}
\addcontentsline{toc}{section}{Contacts}

\begin{itemize}
\tightlist
\item
  Jonathan Emery:
  \href{mailto:jonathan.emery@northwestern.edu}{\nolinkurl{jonathan.emery@northwestern.edu}}
\item
  Chris diaz:
  \href{mailto:chris-diaz@northwestern.edu}{\nolinkurl{chris-diaz@northwestern.edu}}
\end{itemize}

\chapter{Banana Bread}\label{banana-bread}

\begin{recipe}[BananaBread]{J. Wolff Banana Bread}{1 loaf}{90 min (30 min prep, 60 min bake)}
\freeform This recipe makes one loaf of banana bread. Bake it in a J. Wolff bread baker --- although other stone breadpans should work, presumably.
\freeform\rule{\textwidth}{0.05pt}

\newstep Preheat oven to 350 \degrees F.
\ingredient[2-3]{}{overripe bananas}

Mash bananas in a mixing bowl until relatively smooth.

\ingredient[\fr34]{cup}{sugar}
\ingredient[1]{tsp}{vanilla}
\ingredient[2]{tbs}{oil (canola or olive)}
\ingredient[\fr13]{cup}{milk)}

Beat sugar, vanilla, olive oil, and milk into the mashed bananas.

\ingredient[1\fr12]{cup}{flour}
\ingredient[\fr13]{tsp}{baking soda}
\ingredient[\fr3/4]{tsp}{salt}
\ingredient[\fr3/4]{tbs}{ground cinnamon}
\ingredient[\fr1/4]{tsp}{nutmeg}

Add flour, baking soda, salt, cinnamon, and nutmeg. Gently mix just to incorporate --- do not overmix.

\ingredient[\fr12]{cup}{chocolate chips or walnuts}

Fold in walnuts or chocolate chips (or both).

\ingredient[\fr13]{cup}{boiling water}

Pour in boiling water and mix until batter is smooth. It may be a bit watery.

\newstep

Grease stoneware baker and pour in batter. Cook for 60-70 min (check frequently) or until a knife inserted into the bread comes out clean.
\end{recipe}

\chapter{Farmers Market Pasta}\label{farmers-market-pasta}

\begin{recipe}[FarmersMarketPasta]{Farmers' Market Pasta}{6 Portions}{30 min}
\freeform Grab some stuff from the farmers' market and make this pasta. Eat it hot or cold.
\freeform\rule{\textwidth}{0.05pt}
\ingredient[12]{oz}{cavatappi}

Cook pasta according to package directions for \textit{al dente}. Drain and transfer pasta to large bow.

\ingredient[1 of 4]{tbs}{olive oil}
\ingredient[2]{cup}{multicolored cherry tomatoes}
\ingredient[4]{clove}{garlic}
\ingredient[1]{tsp}{kosher salt}
\ingredient[\fr12]{tsp}{ground black pepper}

Heat 1 tbs oil in a large skillet at medium heat. Add tomatoes, garlic, salt, and pepper. Cook, stirring often, until tomatoes begin to soften (2-3 min).  

\ingredient[1]{}{zucchini}
\ingredient[1]{}{red onion, wedged}

Add zucchini and onion and continue cooking until tomatoes burst and zucchini is almost tender (3-4 min).

\ingredient[1]{cup}{corn}

Finally, add corn and cook, stirring constantly, for 1 additional minute.

\ingredient[2]{cups}{packed arugula}
\ingredient[1]{cup}{fresh basil, torn}
\ingredient[3 of 4]{tbs}{olive oil}

Add tomato mixture to posta. Toss with arugula, basil, and remaining olive oil.

\ingredient[\fr12]{cup}{shaved parmesan}

Top with parmesan cheese and serve warm or cold.


\freeform\rule{\textwidth}{0.05pt}

\freeform\rule{\textwidth}{0.05pt}
\textit{nomnomnomnomnom}!
\end{recipe}

\%From RealSimple

\chapter{Moroccan Lamb (or Turkey)
Stew}\label{moroccan-lamb-or-turkey-stew}

\begin{recipe}[MoroccanStew]{Moroccan Lamb (or Turkey) Stew}{4 Portions}{1 hour}
\freeform One of my absolute favorite meals.
\freeform\rule{\textwidth}{0.05pt}
\ingredient[1]{lb}{lamb or turkey}

In a well-oiled pan, saut\'{e} meat until over medium-high heat until cooked through. Set cooked meat aside. Deglaze pan if you like, or if meat was fatty.

\ingredient[2]{cups}{onion}
\ingredient[\fr12]{cup}{carrots}

Add oil to pan. Chop onions vertically and carrots diagonally. Add to pan and saut\'{e} for 4-5 minutes on medium heat, or until soft.

\ingredient[\fr34]{tps}{cumin}
\ingredient[\fr34]{tps}{cinnamon}
\ingredient[\fr12]{tps}{coriander}
\ingredient[\fr14]{tps}{red pepper flakes}

Add coriander, cumin, red pepper flakes, and cinnamon to vegetables, quickly mixing in spices. Saut\'{e} for 30 seconds, stirring constantly.

\ingredient[]{}{Reserved lamb/turkey}
\ingredient[2]{cups}{broth}
\ingredient[\fr12]{cup}{golden raisins}
\ingredient[3]{tbs}{tomato paste}
\ingredient[1\fr12]{tbs}{grated lemon rind}
\ingredient[\fr14]{tps}{salt}
\ingredient[15]{oz}{chick peas (rinsed, drained)}

Add reserved meat, broth, raisins, tomato paste, lemon rind, salt, and chick peas. Bring to a boil, then reduce heat and simmer for 4 minutes (or until mixture thickens). Remove from heat

\ingredient[\fr12]{cup}{chopped, fresh cilantro}
\ingredient[1]{tbs}{fresh lemon juice}

Stir in cilantro and lemon juice.


%\freeform\rule{\textwidth}{0.05pt}\newpage
\freeform\rule{\textwidth}{0.05pt}
%\newstep

%\freeform\rule{\textwidth}{0.05pt}
%Serve with couscous or some grain and fresh bread. Yummy!
\end{recipe}

\chapter{Orzo Salad}\label{orzo-salad}

\begin{recipe}{Spinach Orzo Salad}{20 Portions}{30 minutes}
\freeform Good to share as a holiday salad!
\freeform\rule{\textwidth}{0.05pt}
\ingredient[16]{oz}{Orzo pasta}

Cook orzo according to directions on package --- slightly al dente is fine as well. Drain and rise with cold water.

\ingredient[9]{oz}{spinach}
\ingredient[\fr34]{cup}{dried cranberries}

Finely chop dried cranberries (craisins) and spinach. Add to orzo in a large serving bowl.

\ingredient[\fr34]{cup}{feta cheese}
\ingredient[\fr34]{cup}{balsamic vinaigrette}
\ingredient[\fr12]{tsp}{dried basil}
\ingredient[\fr14]{tsp}{white pepper, ground}

Add these ingredients to the mix and toss so that greens are well-coated with the vinaigrette. Flavored vinaigrette is a nice touch (e.g., strawberry). Refrigerate if not serving immediately.

\ingredient[\fr14]{cup}{sunflower seeds}

Toss in sunflower seeds prior to serving.

%\freeform\rule{\textwidth}{0.05pt}\newpage
\freeform\rule{\textwidth}{0.05pt}
%\newstep

\freeform Jon's recipe is the best!!!

%\freeform\rule{\textwidth}{0.05pt}
%Serve with couscous or some grain and fresh bread.
\end{recipe}

\chapter{Creamy Turkey Noodle Soup}\label{creamy-turkey-noodle-soup}

\begin{recipe}{Creamy Turkey Noodle Soup}{6 Servings}{30 minutes} %{Title}{Servings}{Prep time}
\freeform A soup that Mimi makes after Thanksgiving.       %Freeform formats a general comment.
\freeform\rule{\textwidth}{0.05pt}                         %Creates a nice horizontal rule.

%1st step ingredients---------------------------------------
    %\ingredient[quantity]{unit}{ingredient}
    %\Ingredient{ingredient and quantity}
\ingredient[3]{cups}{turkey}

%1st step action
Use leftover Thanksgiving turkey (or bake). Set aside.

%2nd step ingredients--------------------------------------
\ingredient[5]{tbsp}{butter}
\ingredient[2]{}{carrots}
\ingredient[1]{}{shallot}
\ingredient[1]{tbsp}{celery ceeds}

%2nd step action:
Saut\'{e} ingredients in butter for 5 minutes.

%3rd step ingredients--------------------------------------
\ingredient[\fr13]{cup}{flour} %\fr means fraction: so \fr13 = 1/3
\ingredient[32]{oz}{chicken broth}
\ingredient[\fr12]{cup}{half and half}
\ingredient[\fr12]{cup}{milk}

%3rd step action:
First stir in flour to saut\'{e}d vegetebles and then add liquids and bring mixture to a boil.

%4th step ingredients--------------------------------------
\ingredient[6]{oz}{Reames brand noodles} 

%4th step action:
Add dry noodles to soup and simmer for about 20 minutes.

Cook orzo according to directions on package --- slightly al dente is fine as well. Drain and rise with cold water.

%5th step ingredients--------------------------------------
\ingredient[\fr14]{tsp}{salt}
\ingredient[\fr14]{tsp}{pepper}

Add salt and pepper to taste.

\freeform\rule{\textwidth}{0.05pt} 

\freeform Good to eat for... weeks?

\end{recipe}

\end{document}
